\documentclass[12pt,a4paper]{article}
%\usepackage{paracol}
%\usepackage{amsmath}
\usepackage{graphicx}
%\usepackage{fancyhdr}
%\pagestyle{fancy}
\usepackage[hmargin=2cm,vmargin=4.5cm]{geometry}
\begin{document}
\begin{center}
\section*{\textbf{\Huge PP END-SEM PROJECT REPORT }}
\section*{\textbf{\Large Objectives}}
\end{center}
\begin{itemize}
\item We learned to develop programs for complex real world problems.
\item We learned to apply good programming practices in their code like comments and indentation.
\item We learned to utilize Debugger and its tools like gdb/gnu for error handling.
\item We learned to demonstrate configuration and usage of different software tools used in industry.
\end{itemize}
\pagebreak

\begin{center}
\section*{\textbf{\Large Function Description}}
\end{center}
\subsection*{Function 1 - Arithmetic\_Progression}
\begin{itemize}
\item Arguements: 1 arguement given of datatype int, that is number of terms.
\item Input: Takes first term and common difference of AP as input inside the function.
\item Return Type: void
\item Output: It prints AP using entered first term, common difference and number of terms.
\end{itemize}

\subsection*{Function 2 - Geometric\_Progression}
\begin{itemize}
\item Arguements: 1 arguement given of datatype int, that is number of terms.
\item Input: Takes first term and common ratio of GP as input inside the function.
\item Return Type: void
\item Output: It prints GP using entered first term, common ratio and number of terms.
\end{itemize}

\subsection*{Function 3 - Fibonacci\_Series}
\begin{itemize}
\item Define: A term is sum of its previous two terms taking first two terms as 1.
\item Arguements: 1 arguement given of datatype int, that is number of terms.
\item Return Type: void
\item Output: It prints Fibonacci series upto number of terms passed in function.
\end{itemize}


\subsection*{Function 4 - Hailstone\_Sequence}
\begin{itemize}
\item Define: If a term is n, if n is even then next term = n/2 and if n is odd next term will be 3*n-1.
\item Arguements: 1 arguement given of datatype int, that is number of terms.
\item Input: It takes first term of the sequence as input inside the function..
\item Return Type: void
\item Output: It prints Hailstone Sequence upto number of terms passed in function.
\end{itemize}

\subsection*{Function 5 - Triangular\_numbers}
\begin{itemize}
\item Define: A triangular number are those numbers that counts objects arranged in a equilateral triangle..
\item Arguements: 1 arguement given of datatype int, that is number of terms.
\item Return Type: void
\item Output: It prints triangular numbers upto number of terms passed in function.
\end{itemize}


\subsection*{Function 6 - Palindromic\_numbers}
\begin{itemize}
\item Define: Numbers which are same if we read it either from the start or from the end are palindromic numbers.
\item Arguements: 2 arguements given of datatype int, that is start and end inclusive of the range in which we want to check palindromic numbers.
\item Return Type: void
\item Output: It prints all the palindromic numbers between the range passed in function.
\end{itemize}

\subsection*{Function 7 - Prime\_numbers}
\begin{itemize}
\item Define: Numbers which do not have any perfect divisors are prime numbers.
\item Arguements: 2 arguements given of datatype int, that is start and end inclusive of the range in which we want to check palindromic numbers.
\item Return Type: void
\item Output: It prints all the prime numbers between the range passed in function.
\end{itemize}

\subsection*{Function 8 - Armtrong\_numbers}
\begin{itemize}
\item Define: A three digit integer is integer such that the sum of the cubes of its digit is equal to the number itself.
\item Arguements: 2 arguements given of datatype int(only for 3 digit numbers), that is start and end inclusive of the range in which we want to check palindromic numbers.
\item Return Type: void
\item Output: It prints all the armstrong numbers between the range passed in function.
\end{itemize}

\subsection*{Function 9 - Sum\_of\_Sq\_Cube}
\begin{itemize}
\item Define: This function prints the sum of square and cube of natural numbers.
\item Arguements:  1 arguement given of datatype int, that is number of terms.
\item Return Type: void
\item Output: It prints sum of sqaure and cube of natural numbers upto number of terms passed in function.
\end{itemize}

\subsection*{Function 10 - Power\_Series}
\begin{itemize}
\item Define: This function prints the powers of entered number from 1 to 10.
\item Arguements: 1 arguement given of datatype int, that is number whose powers have to printed.
\item Return Type: void
\item Output: It prints powers from 1 to 10 of the number passed in the function.
\end{itemize}

\pagebreak
\begin{center}
\section*{\textbf{\Large Codes}}
Code screenshots and Output in JAVA
\begin{figure}[p]
\includegraphics[scale=0.7]{JAVA C1}
\end{figure}
\begin{figure}[]
\includegraphics[scale=1.2]{JAVA C2}
\end{figure}
\begin{figure}[]
\includegraphics[scale=1.2]{JAVA C3}
\end{figure}
\begin{figure}[]
\includegraphics[scale=1.2]{JAVA C4}
\end{figure}
\begin{figure}[]
\includegraphics[scale=1.2]{JAVA COutput}
\end{figure}
\end{center}


\pagebreak
\begin{center}
\section*{\textbf{\Large Codes}}
Code screenshots and Output in C++
\begin{figure}[]
\includegraphics[scale=0.7]{cpp1}
\end{figure}
\begin{figure}[]
\includegraphics[scale=1.2]{cpp2}
\end{figure}
\begin{figure}[]
\includegraphics[scale=1.2]{cpp3}
\end{figure}
\begin{figure}[]
\includegraphics[scale=1.2]{cpp4}
\end{figure}
\begin{figure}[]
\includegraphics[scale=1.2]{cppOutput}
\end{figure}
\end{center}

\pagebreak
\pagebreak
\begin{center}
\section*{\textbf{\Large Profile Report}}
Profile Report Screenshots
\begin{figure}[]
\includegraphics[scale=0.7]{pr 1}
\end{figure}
\begin{figure}[]
\includegraphics[scale=1.2]{pr 2}
\end{figure}
\begin{figure}[]
\includegraphics[scale=1.2]{pr 3}
\end{figure}
\begin{figure}[!htb]
\includegraphics[scale=1.2]{pr 4}
\end{figure}
\end{center}


\begin{figure}[]
\includegraphics[scale=0.7]{debug1}
\end{figure}
\begin{figure}[]
\includegraphics[scale=1.2]{debug2}
\end{figure}

\end{document}
